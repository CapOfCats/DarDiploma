\section*{ВВЕДЕНИЕ}
\addcontentsline{toc}{section}{ВВЕДЕНИЕ}
Суда с древних времён служили одним из передовых способов переправки живых и неживых грузов на дальние расстояния по воде. Со временем, новые суда увеличивались в размерах, а следовательно, возрастала сложность управления и контроля этого вида транспорта.

Круизное судно - очень большая и сложная туристическая услуга, представляющая собой путешествие по морю из одной точки в другую и включающая в себя набор услуг, обычно предоставляемый отелями.

Изобретение круизного вида судна повлекло за собой необходимость увеличения количества людей, необходимых для поддержания стабильной и безопасной эксплуатации судна, обеспечения безопасности перевозимого груза и пассажиров в опасной морской среде.

Актуальность темы <<Программное обеспечение для системы контроля и управления доступом на круизном судне>> заключается в необходимости установки порядка и правил на объекте с большим количеством людей для их безопасности, минимизации влияния ручного управления на систему, а также сокращении расходов на персонал, занимающий потенциальные места для продажи.

Система контроля и управления доступом (СКУД) является программно-аппаратным комплексом, в основе которого лежит принцип автоматического определения и реализации прав доступа на охраняемом объекте.

\emph{Цель настоящей работы} – создание приложения для системы контроля и управления доступом на круизном судне. Для достижения поставленной цели необходимо решить \emph{следующие задачи:}
\begin{itemize}
\item исследовать предметную область;
\item спроектировать базу данных;
\item создать базу данных;
\item заполнить базу данных информацией.
\item разработать  интерфейс;
\item реализовать функции приложения;
\end{itemize}

\emph{Структура и объем работы.} Отчет состоит из введения, 4 разделов основной части, заключения, списка использованных источников, 2 приложений. Текст выпускной квалификационной работы равен \formbytotal{lastpage}{страниц}{е}{ам}{ам}.

\emph{Во введении} сформулирована цель работы, поставлены задачи разработки, описана структура работы, приведено краткое содержание каждого из разделов.

\emph{В первом разделе} рассмотрены назначение и важность роли СКУД в наши дни, способы управления доступом, элементы современных СКУД, проанализирована выборка заинтересованных лиц и перспектив развития подобных систем в будущем. Также, в этом разделе представляется сценарий проекта, на котором базируются бизнес-правила будущего программного продукта.

\emph{Во втором разделе} сформулированы основные требования к разработке приложения. В нем содержатся цели и задачи проекта, а также требования пользователя к интерфейсу. В разделе описано полное строение ER-модели базы данных. Кроме того, данный раздел охватывает различные функциональные аспекты, обсуждает архитектуру системы, приводит примеры вариантов использования.

\emph{В третьем разделе} объясняется выбор используемых технологий проектирования, таких как tkinter, customtkinter, SQLite и asynсio. Также в этом разделе осуществляется проектирование пользовательского интерфейса и предоставление описания классов системы.

\emph{В четвертом разделе} данной работы представлен подробный список классов и их методов, которые были использованы в процессе разработки сайта. Кроме того, в этом разделе проводится детальное тестирование разработанного приложения, с целью проверки его функциональности и стабильности.

\emph{В заключении} преподносятся основные выводы и результаты, полученные в ходе разработки проекта.

В приложении А представлен графический материал.
В приложении Б представлены фрагменты исходного кода. 
