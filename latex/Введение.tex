\section*{ВВЕДЕНИЕ}
\addcontentsline{toc}{section}{ВВЕДЕНИЕ}

История контроля доступа берёт своё начало со времён до Нашей Эры. На протяжении всей истории человечества мы наблюдаем за тем, как в нем используются элементы безопасности. Механические деревянные замки были обнаружены на территории современного Ирака еще в 4000 году до н. э., а гробница фараона Тутанхамона была заперта с помощью веревочного узла. Исторически контроль доступа включал в себя нечто большее, чем просто замки. У стражников королевства были сложные смены, которые позволяли им находиться в определенных местах только в определенное время, чтобы обеспечить круглосуточную охрану. В то время такие устаревшие технологии, как рвы, разводные мосты и сторожевые башни были самыми современными инженерными решениями, используемыми для обеспечения защиты периметра и контролируемого доступа. В современном Мире одним из самых эффективных наследников способов контроля является СКУД.

Система контроля и управления доступом (СКУД) является программно-аппаратным комплексом, в основе которого лежит принцип автоматического определения и реализации прав доступа на охраняемом объекте.

В наше время контроль доступа является сложным процессом, применяющимся для защиты мест, имущества, данных или граждан от личных до огромных корпоративных масштабов.  Современные достижения в области технологий позволили создать более надежные и эффективные способы управления и защиты, и современные решения по контролю доступа выходят далеко за рамки стандартных ключей и карт доступа. 

Удаленное управление остается важнейшей задачей с начала 2020 года, позволяя предприятиям обеспечивать безопасность своих зданий, даже когда там никого нет, и прокладывая путь к продуктивным гибридным моделям работы. Удаленный доступ и управление безопасностью позволяют как корпоративным, так и небольшим организациям оставаться гибкими, позволяя командам выполнять повседневные задачи без необходимости физического присутствия на объекте. Такие функции, как удаленное отпирание дверей, полезны для того, чтобы впустить в здание поставщиков, подрядчиков и сотрудников, которые забыли или потеряли свои учетные данные. Благодаря доступу к отчетам о деятельности в режиме реального времени удаленное управление также позволяет организациям гибко перестраиваться на ходу. 

\emph{Цель настоящей работы} – проектирование базы данных круизного лайнера и разработка приложения для СКУД карт доступа пассажиров. Для достижения поставленной цели необходимо решить \emph{следующие задачи:}
\begin{itemize}
\item исследовать предметную область;
\item спроектировать базу данных;
\item создать базу данных;
\item заполнить базу данных информацией.
\item разработать  интерфейс;
\item реализовать функции приложения;
\end{itemize}

\emph{Структура и объем работы.} Отчет состоит из введения, 4 разделов основной части, заключения, списка использованных источников, 2 приложений. Текст выпускной квалификационной работы равен \formbytotal{page}{страниц}{е}{ам}{ам}.

\emph{Во введении} сформулирована цель работы, поставлены задачи разработки, описана структура работы, приведено краткое содержание каждого из разделов.

\emph{В первом разделе} рассмотрены назначение и важность роли СКУД в наши дни, способы управления доступом, элементы современных СКУД, проанализирована выборка заинтересованных лиц и перспектив развития подобных систем в будущем. Также, в этом разделе представляется сценарий проекта, на котором базируются бизнес-правила будущего программного продукта.

\emph{Во втором разделе} сформулированы основные требования к разработке приложения. В нем содержатся цели и задачи проекта, а также требования пользователя к интерфейсу. В разделе описано полное строение ER-модели базы данных. Кроме того, данный раздел охватывает различные функциональные аспекты, обсуждает архитектуру системы, приводит примеры вариантов использования.

\emph{В третьем разделе} объясняется выбор используемых технологий проектирования, таких как tkinter, customtkinter, SQLite и asynсio. Также в этом разделе осуществляется проектирование пользовательского интерфейса и предоставление описания классов системы.

\emph{В четвертом разделе} данной работы представлен подробный список классов и их методов, которые были использованы в процессе разработки сайта. Кроме того, в этом разделе проводится детальное тестирование разработанного приложения, с целью проверки его функциональности и стабильности.

\emph{В заключении} преподносятся основные выводы и результаты, полученные в ходе разработки проекта.

В приложении А представлен графический материал.
В приложении Б представлены фрагменты исходного кода. 
