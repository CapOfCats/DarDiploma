\section*{ОБОЗНАЧЕНИЯ И СОКРАЩЕНИЯ}

БД -- база данных.

ИС -- информационная система.

ИТ -- информационные технологии. 

UI - User Interface, пользовательский интерфейс.

ООП - объектно-ориентированное программирование.

ПО -- программное обеспечение.

РП -- рабочий проект.

СУБД -- система управления базами данных.

ТЗ -- техническое задание.

ТП -- технический проект.

CRUD - Create, Read, Update, Delete, основные операции для работы с данными.

ER (Entity-Relationship model) -- модель данных, позволяющая описывать концептуальные схемы предметной области. ER-модель используется при высокоуровневом (концептуальном) проектировании баз данных.

СКУД -- система контроля и управления доступом

УД -- учётные данные

ЧС -- чрезвычайная ситуация

PkID(Primary key Identifier, первичный ключ) -- в реляционной модели данных один из потенциальных ключей отношения, выбранный в качестве основного ключа (или ключа по умолчанию).

UID(Unique identifier, уникальный идентификатор) -- идентификатор, который используют для однозначного определения объекта, однако признак уникальности не заложен в ID.

FKID(Foreign key identifier, внешний ключ) -- идентификатор, который применяется для принудительного установления связи между данными в двух таблицах с целью контроля данных, которые могут храниться в таблице внешнего ключа.

TreeView -- элемент графического интерфейса для иерархического отображения информации. Представляет собой совокупность связанных отношениями структуры пиктограмм в иерархическом древе.