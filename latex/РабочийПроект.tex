\section{Рабочий проект}
\subsection{Описание классов системы}

\begin{itemize}
	\item \textquotedbl main \textquotedbl является фундаментальным классом для главного меню. Помимо основных свойств, таких как connection, экземпляров utils, controller и tables, содержит в себе стартовые виджеты.
	\item \textquotedbl Controller \textquotedbl -- расширяет \textquotedbl main \textquotedbl, представляя панель управления таблицами, необходимую для взаимодействия пользователя с данными внутри таблицы.
	\item \textquotedbl Utils \textquotedbl применяется для отправки запросов в БД. Содержит только самые необходимые методы, взаимодействующие с БД непосредственно.
	\item \textquotedbl Tables \textquotedbl служит для формирования необходимых сложных запросов к БД. Сформировав строку, класс использует \textquotedbl Utils \textquotedbl для отправки.
	\item \textquotedbl Validator \textquotedbl представляет собой служебный класс, конфигурирующий интерфейс таким образом, чтобы пользователь не мог ввести данные неподходящего формата.
\end{itemize}

Можно выделить следующий список классов, их полей методов, использованных при разработке приложения (таблица \ref{class:table}).

\renewcommand{\arraystretch}{0.8} % уменьшение расстояний до сетки таблицы
\begin{xltabular}{\textwidth}{|X|p{2.5cm}|>{\setlength{\baselineskip}{0.7\baselineskip}}p{4.85cm}|>{\setlength{\baselineskip}{0.7\baselineskip}}p{4.85cm}|}
\caption{Описание классов, используемых в приложении\label{class:table}}\\
\hline \centrow \setlength{\baselineskip}{0.7\baselineskip} Название класса & \centrow \setlength{\baselineskip}{0.7\baselineskip} Модуль, к которому относится класс & \centrow Описание поля/метода & \centrow Поле/метод \\
\hline \centrow 1 & \centrow 2 & \centrow 3 & \centrow 4\\ \hline
\endfirsthead
\caption*{Продолжение таблицы \ref{class:table}}\\
\hline \centrow 1 & \centrow 2 & \centrow 3 & \centrow 4\\ \hline
\finishhead
main & Главный модуль & isES -- поле для определения режима работы программы (Обычный или ЧС) & bool isES\\
\hline  & Главный модуль & window -- экземпляр класса customtkinter.CTk(), представляющий окно & CTk window\\
\hline  & Главный модуль & frame -- экземпляр класса customtkinter.CTkFrame(), представляющий поле для виджетов & CTkFrame frame\\
\hline  & Главный модуль & connection -- строка, содержащая путь для подключения к БД & string connection\\
\hline  & Главный модуль & ask\underline{ }lb, timer\underline{ }lb -- виджеты текста в главном меню &
CTkLabel ask\underline{ }lb

CTkLabel timer\underline{ }lb
\\
\hline  & Главный модуль & pns\underline{ }btn, rms\underline{ }btn, drs\underline{ }btn, psr\underline{ }btn, psr\underline{ }ua\underline{ }btn -- виджеты кнопок в главном меню &
CTkButton pns\underline{ }btn

CTkButton rms\underline{ }btn

CTkButton drs\underline{ }btn

CTkButton psr\underline{ }btn

CTkButton psr\underline{ }ua\underline{ }btn
\\
\hline  & Главный модуль & on\underline{ }closing -- метод для вывода вопроса о том, уверен ли пользователь в своём решении выйти & on\underline{ }closing() Возвращает: ничего\\
\hline Controller & Главный модуль & styles\underline{ }init -- метод для инициализации стилей и добавления его в список ttk.element\underline{ }names & styles\underline{ }init(st). Принимает на вход экземпляр класса Style, который будет настраивать. 

Возвращает: ничего\\
\hline  & Главный модуль & show\underline{ }table -- метод для перехода к панели управления выбранной таблицей. Отображает, настраивает и группирует каждый элемент по сетке & show\underline{ }table(which, window, tables, connection). 

Принимает на вход название таблицы, экземпляр окна, экземпляр tables, и строку подключения connection. 

Возвращает: ничего\\
\hline  & Главный модуль & fields\underline{ }creation -- метод, вызываемый в show\underline{ }table и дополняющий его. Определяет и направленно настраивает поля ввода информации & fields\underline{ }creation(which, frameEx, window). 

На вход принимает название таблицы, экземпляр поля для виджетов, экземпляр окна.

Возвращает: tuple (validatedTFs, combolist) -- кортеж сформированных полей ввода\\
\hline  & Главный модуль & clear -- метод, очищающий поля ввода & clear(keyLabel, combinedControls, window). 

На вход принимает таблицу отображения Id текущей строки для очищения, кортеж полей ввода, экземпляр окна. 

Возвращает: ничего\\
\hline  & Главный модуль & TreeCreate -- метод, отображающий текущую таблицу в виде виджета древа & TreeCreate(tree, table, tableWin, connection). 

На вход принимает экземпляр класса ttk.TreeView, который будет пересоздавать, название таблицы, экземпляр поля виджетов, строку подключения. 

Возвращает: TreeView.tree -- созданный виджет древа\\
\hline  & Главный модуль & TreeRefresh -- метод, обновляющий виджет древа & TreeRefresh(tree, table, connection). 

На вход принимает экземпляр класса ttk.TreeView, который будет обновлять, название таблицы, строку подключения. 

Возвращает: Ничего\\
\hline  & Главный модуль & MoveTo -- метод для перехода просмотра к конкретному элементу таблицы по индентификатору & MoveTo(id, table, combinedControls, currentLb, window, connection). 

На вход принимает идентификатор элемента для перехода, имя таблицы, кортеж полей вывода, табличку отображения текущего Id, экземпляр окна и строку подключения. 

Возвращает: Ничего\\
\hline Utils & Главный модуль & create\underline{ }connection -- метод создаёт подключение к базе данных & create\underline{ }connection(path). 

На вход принимает строковый путь к файлу базы данных. 

Возвращает: sqlite3.connection connection - экземпляр созданного подключения\\
\hline  & Главный модуль & read\underline{ }single\underline{ }row -- считывает единственную строку с БД & read\underline{ }single\underline{ }row(id, connection, table). 

На вход принимает идентификатор, по которому будет считываться строка, экземпляр подключения, название таблицы 

Возвращает: строку в формате кортежа\\
\hline  & Главный модуль & execute\underline{ }query -- посылает команду в БД & execute\underline{ }query(connection, query).

На вход принимает экземпляр подключения и строку команды для выполнения.

Возвращает: bool True/bool False -- флаг о том, успешно ли выполнен запрос\\
\hline  & Главный модуль & execute\underline{ }read\underline{ }query -- отдельный метод для послания запроса в БД на чтение & execute\underline{ }read\underline{ }query(connection, query).

На вход принимает экземпляр подключения и строку команды для выполнения.

Возвращает: строку в формате кортежа\\
\hline  & Главный модуль & timetick -- асинхронный метод для отображения и вывода текущего времени в главном меню & timetick(timerLb).

На вход принимает, куда будет выводиться время.

Возвращает: ничего\\
\hline  & Главный модуль & timetick -- асинхронный метод для отображения и вывода текущего времени в главном меню & timetick(timerLb).

На вход принимает, куда будет выводиться время.

Возвращает: ничего\\
\hline  & Главный модуль & asyncMLoop -- метод для обеспечения асинхронной работы окна с часами & asyncMLoop(wndw).

На вход принимает окно главного меню.

Возвращает: ничего\\
\hline  & Главный модуль & asyncStart -- асинхронный метод для настройки взаимодействия главного меню с часами в асинхронном порядке и обновления полей со временем в БД & asyncStart(window, timer\underline{ }lb, connection).

На вход принимает экземпляр окна, поле для вывода времени, строку подключения к БД.

Возвращает: ничего\\
\end{xltabular}
\renewcommand{\arraystretch}{1.0} % восстановление сетки

\subsection{Модульное тестирование разработанного web-сайта}

Модульный тест для класса User из модели данных представлен на рисунке \ref{unitUser:image}.

\begin{figure}[ht]
\begin{lstlisting}[language=Python]
from django.test import TestCase
from .models import *
User = get_user_model()


class ShpoTestCases(TestCase):

    def setUp(self) -> None:
        self.user = User.objects.create(username='testtestovich', password='testtestovich', first_name='Sad', last_name='')

    def test_2(self):

        self.assertEqual(self.user.first_name, 'Sad')
        self.assertEqual(self.user.last_name, 'Cat')
        print((self.user))
        print((self.user.first_name))
        print((self.user.last_name))
\end{lstlisting}  
\caption{Модульный тест класса User}
\label{unitUser:image}
\end{figure}

\subsection{Системное тестирование разработанного web-сайта}

%На рисунке \ref{main:image} представлена главная страница сайта «Русатом – Аддитивные технологии».
%\newpage % при необходимости можно переносить рисунок на новую страницу
%\begin{figure}[H] % H - рисунок обязательно здесь, или переносится, оставляя пустоту
%\center{\includegraphics[width=1\linewidth]{main1}}
%\center{\includegraphics[width=1\linewidth]{main2}}
%\center{\includegraphics[width=1\linewidth]{main3}}
%\caption{Главная страница сайта «Русатом – Аддитивные технологии»}
%\label{main:image}
%\end{figure}

%На рисунке \ref{menu:image} представлен динамический вывод заголовков, включающий в себя искомые фразы при поиске фраз.

%\begin{figure}[ht]
%\center{\includegraphics[width=1\linewidth]{menu}}
%\caption{Динамический вывод заголовков}
%\label{menu:image}
%\end{figure}

%На рисунке \ref{enter:image} представлен ввод данных для публикации новости.

%\begin{figure}[ht]
%\center{\includegraphics[width=1\linewidth]{enter}}
%\caption{Ввод данных для публикации очень-очень длинной, интересной и полезной новости}
%\label{enter:image}
%\end{figure}
