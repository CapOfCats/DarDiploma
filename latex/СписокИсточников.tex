\addcontentsline{toc}{section}{СПИСОК ИСПОЛЬЗОВАННЫХ ИСТОЧНИКОВ}

\begin{thebibliography}{9}
	
	\bibitem{dictionary}Словарь данных и система управления базами данных / Data Dictionary and Database Management System, 2014.  Автор: Уфимский Государственный Авиационный Технический Университет  – Текст~: электронный. Дата обращения: 02.03.2024. URL: https://studfile.net/preview/942920/page:17/ 
	\bibitem{oopscdb}Бойко И. Объектно-ориентированные СУБД / Автор: И. Бойко. – Киев : Высшая школа, 2014. –- 398 с. -- Текст~: непосредственный. Дата обращения: 05.03.2024. 
	\bibitem{security}Охранные системы и технические средства физической защиты.  /  Security Focus 2022 / Автор: Рыкунов В. –- 398 с. – ISBN: 978-5-9901176-3-1. – Текст~: непосредственный. Дата обращения: 10.03.2024.
	\bibitem{micacc}Проектирование баз данных. СУБД Microsoft Access: Учебное пособие для вузов. / Автор: Н.Н. Гринченко и др. - М.: РиС, 2013. -- 240 c. -- ISBN: 978-5-9912-0295-4.  Текст~: непосредственный. Дата обращения: 14.03.2024.
	\bibitem{dbrealizt}Базы данных. Проектирование, реализация и сопровождение. Теория и практика. / Авторы: Т. Коннолли. - М.: Вильямс И.Д., 2017. - 1440 c. -- ISBN:  978-5-8459-2020-1. Текст~: непосредственный. Дата обращения: 19.03.2024.
	\bibitem{acshistory}From Keys to Credentials: The History of Access Control / Автор: scDataCom, 2024. – Текст~: электронный. Дата обращения: 27.03.2024. URL: https://www.scdatacom.net/blog/from-keys-to-credentials-the-history-of-access-controlnbsp
	\bibitem{dbproject}Введение в проектирование баз данных. / Автор: В.Н. Лукин. - М.: Вузовская книга, 2015. -- 144 c. -- ISBN: 978-5-9502-0761-7. Текст~: непосредственный. Дата обращения: 02.04.2024.	
	\bibitem{sqlite}Medium.com / «SQLite – Как организовывать таблицы» / Автор: А.Шагин, 2020. – Текст~: электронный. Дата обращения: 06.04.2024. URL: https://medium.com/nuances-of-programming/sqlite-как-организовывать-таблицы-81cce38af5b2
	\bibitem{umldb}Проектирование баз данных и UML. / Автор: Р.Д. Мюллер; Пер. с англ. Е.Н. Молодцова. - М.: Лори, 2013. -- 420 c. ISBN: 978-5-85-582322-6. Текст~: непосредственный. Дата обращения: 09.04.2024.
	\bibitem{mobile}Mobile access control guide: 2024./ Автор: kisi  – Текст~: электронный. URL: https://www.getkisi.com/guides/mobile-access-control-guide
	\bibitem{pythonprog} Программирование на Python в примерах и задачах/ Автор: Васильев А.Н. / Бомбора 2023. -- 616с. --ISBN 978-5-04-103199-2 Текст~: непосредственный. Дата обращения: 12.04.2024.
	\bibitem{pythonprof}Python. Чистый код для продолжающих / Автор: Свейгард Э. / Питер, 2023 – 384 с. –- ISBN 978-5-4461-1852-6. -– Текст~: непосредственный. Дата обращения: 17.04.2024.
	\bibitem{cards}Locked. Secured. Protected. Learning About Access Control Systems: Control Cards /Автор: Richmond Security  2024.  – Текст~: электронный. Дата обращения: 25.04.2024. URL: https://www.richmondsecurity.com/learning-about-access-control-systems-control-cards/
	\bibitem{OOP5}Объектно-ориентированный подход. 5-е межд. изд. / Автор: Вайсфельд М./ Питер, 2024 –- 256 с. –- ISBN 978-5-4461-1431-3. – Текст~: непосредственный. Дата обращения: 30.04.2024.
	\bibitem{acsspec}Руководство по составлению спецификаций на СКУД / Автор: Британская ассоциация индустрии безопасности./ Второе переиздание,тираж 500 2014. -- 170 c. Текст~: непосредственный. Дата обращения: 02.05.2024.
	\bibitem{acswork} How do Access Control Systems work?/ Автор: Aatel 2024. – Текст~: электронный. Дата обращения: 07.05.2024. URL: https://www.aatel.com/portfolio/how-do-access-control-systems-work/	
	\bibitem{acschronicle}Chronicling the Evolution of Access Control Credentials/Автор: Jim Dearing, hidglobal, 2021. – Текст~: электронный. Дата обращения: 10.05.2024. URL: https://blog.hidglobal.com/2021/03/chronicling-evolution-access-control-credentials
	\bibitem{acs}Системы контроля и управления доступом. / Авторы: Ворона В.А.,Тихонов В.А. / Горячая Линия-Телеком 2018 г. –- 272 с. ISBN:978-5-9912-0059-2. Текст~: непосредственный. Дата обращения: 14.05.2024.
	\bibitem{jtkinter}Documentation: tkinter / Python interface to Tcl/Tk, 2023/ Автор: python.org. – Текст~: электронный. Дата обращения: 19.05.2024. URL: https://docs.python.org/3/library/tkinter.html
	\bibitem{async} Asyncio и конкурентное программирование на Python / ДМК Пресс, 2023/ Автор: Фаулер М. –- 398 с. –- ISBN 978-5-93700-166-5. – Текст~: непосредственный.  Дата обращения: 23.05.2024.
	
\end{thebibliography}
