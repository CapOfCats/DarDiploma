\addcontentsline{toc}{section}{СПИСОК ИСПОЛЬЗОВАННЫХ ИСТОЧНИКОВ}

\begin{thebibliography}{9}
	
	\bibitem{dictionary}Словарь данных и система управления базами данных / Data Dictionary and Database Management System, 2019. – Текст~: электронный. URL: https://studfile.net/preview/942920/page:17/
	\bibitem{oopscdb}Бойко И. Объектно-ориентированные СУБД / И. Бойко. – Киев : Высшая школа, 2014. –- 398 с. -- Текст~: непосредственный.  
	\bibitem{security}Рыкунов В. Охранные системы и технические средства физической защиты.  /  Security Focus 2022. –- 398 с. – ISBN: 978-5-9901176-3-1. – Текст~: непосредственный.   
	\bibitem{micacc}Гринченко, Н.Н. Проектирование баз данных. СУБД Microsoft Access: Учебное пособие для вузов. / Н.Н. Гринченко и др. - М.: РиС, 2013. -- 240 c. -- ISBN: 978-5-9912-0295-4.  Текст~: непосредственный.
	\bibitem{dbrealizt}Коннолли, Т. Базы данных. Проектирование, реализация и сопровождение. Теория и практика. / Т. Коннолли. - М.: Вильямс И.Д., 2017. - 1440 c. -- ISBN:  978-5-8459-2020-1. Текст~: непосредственный.
	\bibitem{acshistory}scDataCom [Электронный ресурс] From Keys to Credentials: The History of Access Control, 2024. – Текст~: электронный. URL: https://www.scdatacom.net/blog/from-keys-to-credentials-the-history-of-access-controlnbsp
	\bibitem{dbproject}Лукин, В.Н. Введение в проектирование баз данных. / В.Н. Лукин. - М.: Вузовская книга, 2015. -- 144 c. -- ISBN: 978-5-9502-0761-7. Текст~: непосредственный.	
	\bibitem{sqlite}Medium.com. / «SQLite – Как организовывать таблицы» Автор:А.Шагин, 2020. – Текст~: электронный. URL: https://medium.com/nuances-of-programming/sqlite-как-организовывать-таблицы-81cce38af5b2
	\bibitem{umldb}Мюллер, Р.Д. Проектирование баз данных и UML. / Р.Д. Мюллер; Пер. с англ. Е.Н. Молодцова. - М.: Лори, 2013. -- 420 c. ISBN: 978-5-85-582322-6. Текст~: непосредственный.
	\bibitem{mobile}kisi [Электронный ресурс] Mobile access control guide: 2024.  – Текст~: электронный. URL: https://www.getkisi.com/guides/mobile-access-control-guide
	\bibitem{pythonprog} Васильев А.Н. Программирование на Python в примерах и задачах/ Бомбора 2023. -- 616с. --ISBN 978-5-04-103199-2 Текст~: непосредственный.    	
	\bibitem{pythonprof}Свейгард Э. Python. Чистый код для продолжающих / Питер, 2023 – 384 с. –- ISBN 978-5-4461-1852-6. -– Текст~: непосредственный.    
	\bibitem{cards}Richmond Security / Locked. Secured. Protected/ Learning About Access Control Systems: Control Cards, 2024.  – Текст~: электронный. URL: https://www.richmondsecurity.com/learning-about-access-control-systems-control-cards/
	\bibitem{OOP5}Вайсфельд М. Объектно-ориентированный подход. 5-е межд. изд. / Питер, 2024 –- 256 с. –- ISBN 978-5-4461-1431-3. – Текст~: непосредственный.    	
	\bibitem{acsspec}Британская ассоциация индустрии безопасности. Руководство по составлению спецификаций на СКУД / Второе переиздание,тираж 500 2014. -- 170 c. Текст~: непосредственный.
	\bibitem{acswork}aatel [Электронный ресурс] How do Access Control Systems work?: 2024. – Текст~: электронный. URL: https://www.aatel.com/portfolio/how-do-access-control-systems-work/	
	\bibitem{acschronicle}hidglobal [Электронный ресурс] Chronicling the Evolution of Access Control Credentials: Jim Dearing, 2021. – Текст~: электронный. URL: https://blog.hidglobal.com/2021/03/chronicling-evolution-access-control-credentials
	\bibitem{acs}Ворона В.А.,Тихонов В.А. Системы контроля и управления доступом. / Горячая Линия-Телеком 2018 г. –- 272 с. ISBN:978-5-9912-0059-2. Текст~: непосредственный.	
	\bibitem{jtkinter}python.org / Documentation: tkinter / Python interface to Tcl/Tk, 2023. – Текст~: электронный. URL: https://docs.python.org/3/library/tkinter.html
	\bibitem{async}Фаулер М. Asyncio и конкурентное программирование на Python / ДМК Пресс, 2023 –- 398 с. –- ISBN 978-5-93700-166-5. – Текст~: непосредственный.  
	
\end{thebibliography}
