\section{Технический проект}
\subsection{Общая характеристика организации решения задачи}

Необходимо спроектировать и разработать приложение, которое обеспечит функционирование СКУД на круизном лайнере AIDABlu.

Приложение представляет собой панель управления для работы с данными в базе данных системы. Панель содержит текстовую и графическую информацию (TreeView).

Приложение является десктопным, т.е. располагается и запускается внутри лишь одной операционной системы. Каждое окно приложения (за исключением главного) – это панель управления для конкретной таблицы базы данных. Приложение разработано на языке Python v3.10. Управление данными БД реализовано с помощью библиотеки sqlite3 и SQL -- языка структурированных запросов к базе данных.

\subsection{Общие сведения о программно-информационной системе}

Полное наименование системы: Программное обеспечение для системы контроля и управления доступом на круизном судне.

Краткое обозначение системы: \textquotedbl СКУД на круизном лайнере \textquotedbl.

Описание системы: \textquotedbl СКУД на круизном лайнере \textquotedbl предназначена для корпораций, организующих круизные путешествия, предоставляя им платформу для удобного контроля и управления данными всей бизнес-системы круизного лайнера на время поездки. Система создана для обеспечения комфортной и безопасной поездки каждого пассажира и функционирования в форс-мажорных ситуациях.

Условия эксплуатации: \textquotedbl СКУД на круизном лайнере \textquotedbl предназначена для использования как в нормальных, так и в чрезвычайных условиях работы.

Архитектура системы: Программное обеспечение основано на десктопной архитектуре, используя современные технологии разработки, включая TKinter и CustomTkinter для Front-End части и sqlite3 для реализации запросов к БД в Back-End. Система использует базу данных SQLite.

Технологии и инструменты: В разработке использовались tkinter, customtkinter, sqlite3, asyncio, time

\subsection{Обоснование выбора технологии проектирования}

\subsubsection{Описание используемых технологий и языков программирования}

В процессе разработки приложения используются программные средства и языки программирования. Каждое программное средство и каждый язык программирования применяется для круга задач, при решении которых они необходимы.

\subsubsection {tkinter}
Выбор tkinter для Front-End обосновывается его простотой и кроссплатформенной поддержкой, а также минимальными требованиями к интерфейсу в пользу быстродействия. 

Пакет tkinter («интерфейс Tk») - это интерфейс Python для создания GUI. Tkinter доступен на большинстве платформ Unix, включая macOS, а также на системах Windows.

Tkinter входит в состав большинства инсталляций Python, что делает его легкодоступным для разработчиков, которые хотят создавать приложения с графическим интерфейсом, не требуя дополнительных инсталляций или библиотек.

\subsubsection {customtkinter}
Выбор customtkinter для Front-End обосновывается его расширенным функционалом по сравнению с tkinter. Customtkinter предлагает большой список параметров для настройки виджетов и является полностью совместимым с элементами tkinter.

CustomTkinter - это библиотека пользовательского интерфейса для настольных компьютеров на основе Tkinter, которая обеспечивает современный вид и полностью настраиваемые виджеты.

\subsubsection {SQLite}
Выбор SQLite в качестве системы управления базами данных также обосновывается его удобством как на этапе проектирования, так и на этапе реализации.

SQLite - это библиотека на языке C, которая предоставляет легкую дисковую базу данных, не требующую отдельного серверного процесса и позволяющую обращаться к базе данных с помощью нестандартного варианта языка запросов SQL. Некоторые приложения могут использовать SQLite для внутреннего хранения данных. Также можно создать прототип приложения с использованием SQLite, а затем перенести код на более крупную базу данных, такую как PostgreSQL или Oracle.

\subsubsection {asyncio и time}
Модули asyncio и time были использованы для актуализации работы системы в реальном времени и многопоточном режиме.


\subsubsection{Язык структурированных запросов к базе данных SQL}
SQL - это стандартизированный язык программирования, который используется для управления реляционными базами данных и выполнения различных операций над данными в них. 

SQL используется для следующего:
\begin{itemize}
	\item изменение структуры таблиц данных и индексов базы данных;
	\item добавление, обновление и удаление строк данных;
	\item извлечение подмножеств информации из реляционных систем управления базами данных (РСУБД).
\end{itemize}


\subsubsection{Язык программирования Python}

\paragraph{Достоинства языка Python}
Python - очень продуктивный язык. Благодаря простоте Python разработчики могут сосредоточиться на решении проблемы. Написание кода экономит время и освобождает его для более ёмкой работы с другими составляющими проекта.

Python поставляется под лицензией OSI с открытым исходным кодом. Это делает его свободным для использования и распространения. Можно загружать исходный код, изменять его и даже распространять свою версию Python. Это полезно для организаций, которые хотят изменить некоторые специфические функции и использовать свою версию для разработки.

Стандартная библиотека Python огромна, в ней можно найти практически все функции, необходимые для решения любой задачи. Таким образом, не придется зависеть от внешних библиотек.

Во многих языках, таких как C/C++, для запуска программы на разных платформах необходимо изменять код. С Python дело обстоит иначе. Вы пишете один раз и запускаете программу в любом месте.

\paragraph{Недостатки языка Python}

Язык программирования Python использует большой объем памяти. Это может быть недостатком при создании приложений, когда предпочтение отдаётся оптимизации памяти.

Python используется для программирования на стороне сервера. Пользователь не видит Python на стороне клиента или в мобильных приложениях.

Python - динамически типизированный язык, поэтому тип данных переменной может измениться в любой момент. Переменная, содержащая целое число, в будущем может стать строкой, что может привести к ошибкам времени выполнения.

\subsection{Проектирование пользовательского интерфейса}
На основании требований к пользовательскому интерфейсу, представленных в пункте 2.3 технического задания, был разработан графический интерфейс десктопного приложения с применением python tkinter, customtkinter и SQLite. Этот процесс подчеркивает важность интуитивно понятного и эффективного взаимодействия с пользователем. Разработанный интерфейс ориентирован на обеспечение легкости в использовании и интуитивного понимания функционала приложения, предоставляя пользователю простое и эффективное взаимодействие с приложением.

1. \textbf{Навигация по таблицам с данными:} Реализация функции навигации на основе полей psr\underline{ }btn, psr\underline{ }ua\underline{ }btn, drs\underline{ }btn, rms\underline{ }btn, pns\underline{ }btn, acs\underline{ }btn;

2. \textbf {Панель управления для каждой из таблиц с данными:} диверсификация интерфейсов происходит на основе метода show\underline{ }table класса Controller;

3. \textbf{Отображение текущей таблицы с данными в реальном времени:} актуальность этого графического элемента (дерева) поддерживается за счёт методов TreeRefresh и TreeCreate класса Controller;

4. \textbf{Воздействие на внесённые данные базы внутри приложения:} Добавление, изменение или удаление элементов реализованы в методах add\underline{ }element, update\underline{ }element и delete\underline{ }element класса Tables соответственно, внутри которых формирование строк запросов к БД реализовано с помощью класса Utils.

5. \textbf{Навигация и лёгкий поиск среди элементов среди данных:} Пользователь может переходить по элементам последовательно (метод MoveTo класса Controller), или же использовать отдельное текстовое поле для метода search\underline{ }element класса Tables, перейдя сразу к искомому элементу таблицы.

6. \textbf{Отображение локального времени:} Учитывая специфику проекта -- систему для круизного лайнера, учтено, что часовой пояс может измениться во время путешествия. На уровне python и SQL была установлена привязка к локальному времени (localtime), а функционал часов запущен в асинхронном потоке во избежание ошибок и помех работы основной системы.

7. \textbf{Отображение информации о доступе выбранного пассажира к выбранной двери:} На основе правил, прописанных в полях Комнаты, к которой привязана выбранная Дверь, в отдельном окне отображается информация о том, есть ли у Пассажира доступ к этой Двери.

\begin{figure} [ht]
	\centering
	\includegraphics[width=1\linewidth]{images/Example2}
	\caption{модели интерфейса <<Панель управления для каждой из таблиц с данными>> и <<Отображение текущей таблицы с данными в реальном времени>>}
	\label{fig:example2}
\end{figure}

Процесс изменения данных максимально упрощён и включает следующие шаги:

1. В главном меню пользователь выбирает поле с названием необходимого ему вида данных.

2. В появившемся окне пользователь находит нужный ему элемент посредством <<Пролистывания>> или посредством поиска по идентификатору, при нажатии на соответствующие подписанные поля.

3. Пользователь изменяет некоторые данные в полях элемента и выбирает опцию <<Изменить>>, если ему нужно было редактировать элемент, или <<Удалить>>, если была необходимость удалить элемент.

4. Если пользователю необходимо добавить элемент, он должен заполнить поля ввода: Наименование, принадлежность, номер, статус, скорость открытия, вместимость (В случае работы с данными о дверях) и выбрать опцию <<Добавить>>.

5. На последнем этапе, если все обязательные поля были заполнены данными корректного формата, пользователь получает сообщение об успешном проведении операции и она выполняется.

\begin{figure} [ht]
	\centering
	\includegraphics[width=1\linewidth]{images/Example1}
	\caption{модель интерфейса изменения данных элемента в данных о дверях}
	\label{fig:example1}
\end{figure}
Процесс получения информации о доступе является еще более простым и состоит из следующих этапов:

1. В главном меню пользователь выбирает опцию <<Проверка доступа>>.

2. В появившемся окне пользователь с помощью переключателя выбирает нужную ему группу поиска среди пассажиров - таблицу <<Пассажиры>> или <<Дети>>.

3. Пользователь вводит идентификаторы пассажира и двери в поля \textquotedbl Пассажир \textquotedbl и \textquotedbl Дверь \textquotedbl соответственно и выбирает опцию <<Проверить доступ>>

4. На последнем этапе, если все обязательные поля были заполнены данными корректного формата и в БД содержатся данные о предоставленном/отказанном доступе, пользователь получает сообщение о том, есть ли у указанного пассажира доступ к указанной двери.

\begin{figure} [ht]
	\centering
	\includegraphics[width=1\linewidth]{images/Example3}
	\caption{модель интерфейса <<отображение информации о доступе выбранного пассажира к выбранной двери>>.}
	\label{fig:example3}
\end{figure}

\subsection{Диаграмма размещения}

Диаграмма размещения, отображаемая на рисунке \ref{fig:commonscheme4}, является фундаментальным инструментом для иллюстрации взаимосвязей между программными и аппаратными компонентами системы. Этот элемент визуализации служит для акцентирования значимости стратегического планирования
в процессе разработки распределенных систем. Детальное и глубокое понимание этих взаимосвязей критически важно для успешного создания и функционирования распределенных информационных систем.

Каждый компонент системы, будь то программный или аппаратный, играет важную роль в обеспечении её общей эффективности и надежности.
Подход, основанный на стратегическом планировании, способствует оптимизации этих взаимодействий и повышает вероятность успешной реализации и эксплуатации системы в целом.


\begin{figure} [ht]
	\centering
	\includegraphics[width=1\linewidth]{images/CommonScheme4}
	\caption{Диаграмма размещения}
	\label{fig:commonscheme4}
\end{figure}

Она является хорошим средством для показа маршрутов перемещения объектов и компонентов в распределенной системе.

\subsection{Описание архитектуры приложения}

Архитектура приложения, реализованная в рамках текущей работы, базируется на модели MVC (Model-View-Controller). Этот паттерн был избран из-за его способности к эффективному распределению функциональных обязанностей между структурными компонентами системы, а также способствованию упрощению процессов разработки и тестирования.

MVC реализован следующим образом:

1. Модель (Model) и Контроллер (Controller): Эти компоненты обеспечивают управление бизнес-логикой и обработку данных, а также связь между пользовательским интерфейсом и базой данных;

2. Представление (View): Реализовано через фронтенд, использующий пакеты tkinter, customtkinter и treeview, и отвечающий за визуализацию информации и интерактивное взаимодействие с пользователем.

Ключевым аспектом в управлении данными является использование системы управления базами данных SQLite. Этот выбор был обусловлен высокой надежностью SQLite, её устойчивостью к атакам типа \textquotedbl SQL-Инъекция \textquotedbl и гибкостью в обработке сложных запросов, что имеет критическое значение для эффективного функционирования бэкенда.

Применение архитектуры MVC принесло следующие ключевые преимущества:

\begin{itemize}
	\item Четкое Разделение Обязанностей: Эффективное разграничение между пользовательским интерфейсом (класс main.py и controller.py) и back-end логикой (tables.py) значительно упрощает процесс разработки и последующей поддержки системы
	\item Гибкость и Масштабируемость: Благодаря MVC, архитектура приложения легко адаптируется и масштабируется, что позволяет разработчикам
	модифицировать или расширять отдельные части системы без влияния на другие;
	\item Упрощение Тестирования: Независимость компонентов архитектуры MVC облегчает процедуру тестирования, позволяя проводить её для каждого элемента в отдельности
\end{itemize}

Такой подход устанавливает предпочтительную форму записи организации функционала и способ его использования в тестировании. Он обеспечивает создание надежной, гибкой и масштабируемой системы, а также применим как в крупных, так и в малых проектах.

Модули:

В backend применяется модуль asyncio для организации работы системного времени в отдельном потоке.

Модуль re включается для работы с регулярными выражениями для задания правил написания, валидации полей.

Модуль sqlite3 используется для организации подключения и запросов к базе данных.

Для создания frontend используются модули tkinter, customtkinter и treeView. TreeView необходим для отображения виджета древа таблицы в реальном времени,  customtkinter - для основного интерфейса приложения, а tkinter - для служебного упорядочивания элементов интерфейса customtkinter и treeview.

Классы:

main.py - в этом классе прописан front-end главного меню с последующей передачей вводимых данных в другие компоненты.

Utils.py - служебный класс, внутри которого реализованы функции запросов к базе данных, асинхронное обновление системного времени(Например, подключение к БД - create\underline{ }connection(), чтение - read\underline{ }single\underline{ }row(), запрос - execute\underline{ }query() и т.д.).

Validator.py - служебный класс для задания правил написания (валидации). Необходим для проверки корректности вводимых данных(Например, метод FKValid() - для валидации внешних ключей, overvalidation() для дополнительной проверки корректности перед валидацией)

Controller.py - в этом классе прописана front-end логика панели управления таблицами данных с последующей передачей вводимых данных в другие компоненты, а также выводом полученных данных на экран.

Tables.py - служебный класс, созданный для формирования запросов к БД. Вся логика взаимодействия с данными внутри базы, таким как добавление - add\underline{ }element(), удаление - delete\underline{ }element() и т.д., содержится в этом классе.